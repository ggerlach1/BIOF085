\documentclass{article}


% Formatting
\usepackage[utf8]{inputenc}
\usepackage[margin=1in]{geometry}
\usepackage[titletoc,title]{appendix}

% Math
\usepackage{amsmath,amsfonts,amssymb,mathtools}

% Images
\usepackage{graphicx,float}

% Tables

\usepackage[ruled,vlined]{algorithm2e}
\usepackage{algorithmic}
\usepackage{hyperref}

% References
\usepackage{biblatex}
\addbibresource{references.bib}

% Title content
\title{Day 3 Morning: Machine Learning}
\author{Gabriella Gerlach}
\date{\today}

\begin{document}

\maketitle
\section{Data munging for ML}
Associated screen cast: \href{https://www.youtube.com/watch?v=3S19iHHfqzA}{link}
Load the fmri data using \texttt{seaborn.load\_datasets("fmri")}
\begin{enumerate}
		\item Remove the column "subject"
		\item Convert the categorical variables to dummies 
		\item Extract "signal as the outcome of interest
\end{enumerate}

\section{Data splitting}
Associated screen cast: \href{https://www.youtube.com/watch?v=mNVH8V5FLeA}{link}
Split the dataset from the previous assignment into a 80/20 training/test split 
\section{fit the model}
Associated screen cast: \href{https://www.youtube.com/watch?v=LL_AQPg3Gfk}{link}
Fit a decision tree and a random forest model to the training data
\section{model performance}
Associated screen cast: \href{https://www.youtube.com/watch?v=VfSIL-Bzz8k}{link}
Determine the performance of the models using mean squared errors and R2
\section{Cross-validation}
Associated screen cast: \href{https://www.youtube.com/watch?v=kUuZzYHbK5I}{link}
Perform 5-fold cross-validation (split into 5 pieces) to better estimate the errors. 
compare to prediction error using test data.

Draw a graph to show how well each model predicted an individual's signal 
\end{document}
